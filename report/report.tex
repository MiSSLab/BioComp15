\documentclass[a4paper,10pt]{article}
\usepackage{amsmath}
\usepackage{amssymb}
\usepackage{pifont}
\usepackage[utf8]{inputenc}
\usepackage[left=2.5cm,right=2.5cm,top=2.5cm,bottom=2.0cm]{geometry}
\usepackage{multirow}
\usepackage{geometry}
\usepackage{pdflscape}
\usepackage{xcolor}
\usepackage{listings}
\lstset{basicstyle=\ttfamily,
  showstringspaces=false,
  commentstyle=\color{red},
  keywordstyle=\color{blue}
}

\newcommand{\xmark}{\ding{55}}%


\title{Elements of Computational Biology\\ \Large
Subject 15: Distance phylogenetics: UPGMA and NJ}
\author{Agata Radys, Paweł Cejrowski, Łukasz Myśliński}
\date{\today}

\begin{document}
\newgeometry{margin=1.8cm}
\maketitle

\section{Algorithms}

\subsection{UPGMA}
\subsection{NJ}
\subsection{Comparing topology}

\section{Usage}
Application can be downloaded from Github and compiled using \texttt{Maven}.
\begin{lstlisting}[language=bash,caption={Building project}]
git clone git@github.com:MiSSLab/BioComp15.git
mvn package
\end{lstlisting}
Created Java archive can be run using \texttt{JRE}. Sample data can be found in directory \texttt{resources/}.
\begin{lstlisting}[language=bash,caption={Running project using data1.matrix}]
java -jar -Dfilename="resources/data1.matrix" \
    target/distance-phylogenetics-jar-with-dependencies.jar
\end{lstlisting}

\section{Data formats}
\subsection{Input}
Application requires \texttt{CSV} data format and quadratic matrix of distances.

\begin{lstlisting}[caption={Example data file content}]
a,b,c,d,e
0,8,8,5,3
8,0,3,8,8
8,3,0,8,8
5,8,8,0,5
3,8,8,5,0
\end{lstlisting}
Labels in header has to be lexicographically sorted, dense vector with every column matching \texttt{"[a-z]+"}.
\subsection{Output}
\end{document}

